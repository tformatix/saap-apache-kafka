\section{MQTT}
\label{cha:mqtt}

MQTT (Message Queuing Telemetry Transport) and Kafka are two essential technologies for managing IoT data. MQTT is a lightweight messaging protocol that is ideal for resource-constrained devices and low-bandwidth networks. Kafka, on the other hand, is a distributed streaming platform that is optimized for high throughput, fault tolerance, and scalability \cite{hugo2020bridging}.

MQTT uses a publish-subscribe model, which makes it easy for devices to share data with each other and with a central service. MQTT messages tend to be small, which minimizes the amount of data that needs to be transmitted over the network. This is important for IoT devices that operate with limited bandwidth \cite{al2020investigating}.

MQTT also emphasizes low latency, which ensures that messages are delivered quickly and reliably. This is important for applications that need real-time data insights \cite{al2020investigating}.

Kafka provides high throughput, which makes it a good choice for processing large volumes of data. It also has fault tolerance features, which ensure that data is not lost in the event of a server outage. Kafka is also scalable, which means that it can be easily expanded to accommodate increasing data volumes \cite{shapira2021kafka}.

When used together, MQTT and Kafka provide a powerful and flexible solution for managing IoT data. MQTT handles the initial data ingestion, while Kafka handles the processing and analysis of the data. This allows for efficient and scalable data management for IoT applications \cite{hugo2020bridging}.